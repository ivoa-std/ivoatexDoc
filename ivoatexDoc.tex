\documentclass[11pt,a4paper]{ivoa}
\input tthdefs

\usepackage{listings}
\lstloadlanguages{sh,make,[latex]tex}
\lstset{flexiblecolumns=true,numberstyle=\small,numbers=left,
  identifierstyle=\texttt,defaultdialect=[latex]tex, language=tex}

\usepackage[utf8]{inputenc}

\definecolor{texcolor}{rgb}{0.4,0.1,0.1}
\newcommand{\texword}[1]{\texttt{\color{texcolor} #1}}

\title{The \ivoatex\ Document Preparation System}

\ivoagroup{Standards and Processes}

\author[http://www.ivoa.net/cgi-bin/twiki/bin/view/IVOA/MarkusDemleitner]{Markus Demleitner}
\author[http://www.ivoa.net/cgi-bin/twiki/bin/view/IVOA/MarkTaylor]{Mark Taylor}
\author[http://www.ivoa.net/cgi-bin/twiki/bin/view/IVOA/PaulHarrison]{Paul Harrison}
\author[http://www.ivoa.net/cgi-bin/twiki/bin/view/IVOA/MarcoMolinaro]{Marco Molinaro}

\editor{Markus Demleitner}

\previousversion{This is the first public release}
       

\begin{document}
\begin{abstract}
This note describes the \ivoatex\ document preparation system for IVOA
standards and notes.  \ivoatex\ supports the production of
PDF and HTML renderings of the documents with sources in
plain text suitable for version control, as is desirable for normative
texts.  This note contains a user guide as well as a discussion of 
\ivoatex's dependencies and its implementation.
\end{abstract}


\section*{Acknowledgments}

\ivoatex\ heavily draws from experiences made with previous markup-based
document preparation systems, in particular LaTeX classes and
infrastructure created by Sebasti\'en Derriere and Mark Taylor, as well
as Paul Harrison's XML-based ivoadoc system.

We thank tth's author, Ian Hutchinson, for generous technical support
and prompt provision of solutions in the upstream source where necessary.

\section{Introduction}

Creating and developing standards is a  big part of the operations of
the International Virtual Observatory Association (IVOA).
As these are normative texts, attention to detail is very important, and
being able to rigorously track changes to the documents is highly
advantageous.  

Standards are also often developed cooperatively, which means that
capabilities for branching and merging are desirable.  This strongly
suggests employing version control systems for document authoring.
Change tracking in software designed for editing office documents, to
the extent it is supported at all, usually requires significant
manual intervention, is optional, often used incorrectly, and frequently
lacks interoperability.  Led by these considerations, it was decided that
\ivoatex\ would have to be based on plain text source files.

As mandated by \citet{std:docSTD}, finished documents have to be at
least available in PDF, while an additional HTML rendering for online
use is recommended.  A document preparation system should thus be
able  to produce documents in these formats in at least acceptable
quality.

With these constraints in mind, several possible solutions were
investigated.  Paul
Harrison's ivoadoc
system\footnote{\url{https://volute.googlecode.com/svn/trunk/projects/ivoapub/ivoadoc}} 
went for XHTML as
an input format and used XSLT2 and XML-FO as document processors.  While
this facilitated several interesting features  -- for instance,
automatic extraction and formatting of XML schema fragments or
straightvorward embedding of RDFa markup for machine-readable examples
--, it turned out that tooling issues were severe (e.g., reliable use
of SGML catalogs\footnote{This is important as retrieval of DTDs and
similar data from their commonly used system identifiers (i.e.,
typically W3C web servers) is at least undesirable and in practice
causes massive delays in formatting due to rate limiting on the part of
the W3C.}, non-free hyphenation patterns, classpath issues) and the use
of XML-FO for PDF generation yielded inferior renderings with little
prospect for improvements by third parties. Also, authors disliked
writing HTML tags.

Other options considered for source languages included docbook or one of
the newer lightweight markup languages (ReStructuredText, markdown,
etc).  In each case, there were concerns either regarding the system's power
and flexibility or its ease of installation and maintenance.

Meanwhile, several documents -- to mention just a few, SAMP, VOTable,
and VOUnits -- had successfully used \TeX-based systems typically
derived from work done in the early 2000s by Sebasti\'en Derriere.
\ivoatex\ essentially is a generalisation of these standards' formatting
systems, also inheriting from them the use of the \texttt{make} tool to
automate workflows.

\ivoatex\ was extended to relieve document editors from some of the
bookkeeping involved with producing IVOA standards and provide authors
with uniform solutions for common problems in standards typesetting.

In the remainder of this document, we give quick-start instructions
on installation and authoring in sect.~\ref{sect:quick} and continue
a more thorough discussion of \ivoatex's facilities with a special focus
on enabling proper automatic production of both HTML and PDF output in
sect.~\ref{sect:authoring}.  In sect.~\ref{sect:impl}, additional
details on the implemenation are given for the benefit of authors
planning to extend \ivoatex.  We close with a discussion of open issues
and desirable developments.

\section{Installation and Quick Start}
\label{sect:quick}

\subsection{Dependencies}

\ivoatex\ is designed to work more or less out of the box on common
POSIX-compliant systems; no non-free software is required for operation.
Its main uncommon dependency, the tth translator, is delivered with the
distribution and built on-demand. It is a relatively compact program
written in highly portable C generated through lex that translates LaTeX
to HTML.
As \ivoatex's tth may at times offer
some enhancements over the upstream tth, using a system-installed tth is
discouraged.

The remaining dependencies include:

\begin{itemize}
\item A \LaTeX\ distribution with some commonly available packages (calc,
graphicx, xcolor, ifthen, doc, paralist, url, natbib, caption, hyperref,
the agsm bibliography style).  It is recommended to install TeXLive.
\item A sufficiently capable implementation of \texttt{make}, with GNU's
implementation recommended.
\item The XSLT1 processor \texttt{xsltproc} (a different processor can
be used, but that would probably require custom make rules).
\item The \texttt{gcc} compiler (another C compiler could be used; the
central makefile should probably be amended to allow easier changes
here).
\item The \texttt{zip} archiver for package generation.
\item \texttt{imageMagick} and \texttt{ghostscript} if vector graphics
is to be processed.
\end{itemize}

On Windows, it is recommended to run \ivoatex\ within cygwin, where all
dependencies can easily be installed from cygwin's repository.  On
Debian-derived systems, the dependencies should be present after
running a distribution-specific adaption of

\begin{lstlisting}[language=sh]
apt-get install build-essential texlive-full zip xsltproc\
  imagemagick ghostscript cm-super
\end{lstlisting}

(cm-super contains vector versions of computer modern fonts in T1
encoding), on RPM-based systems something like

\begin{lstlisting}[language=sh]
yum install texlive-scheme-full libxslt make gcc zip\
  ImageMagick ghostscript
\end{lstlisting}

should pull in everything that is necessary.

With Mac OS X, a convenient way to obtain the dependencies is to install
MacPorts\footnote{\url{https://www.macports.org/}} and then run

\begin{lstlisting}[language=sh]
port install ImageMagick  libxslt ghostscript  texlive +full
\end{lstlisting}

To see if the prerequisites are there and compatible with \ivoatex, try
building this document from its volute source:

\begin{lstlisting}[language=sh]
svn co https://volute.googlecode.com/svn/trunk/projects/ivoapub/ivoatexDoc
cd ivoatexDoc
make biblio   # update the bibliography
make forcetex # make a PDF ignoring timestamps
make ivoatexDoc.html  # make an html
make package # make a zipfile for IVOA submission
\end{lstlisting}

Messages about unexpected end tags in HTML generation are
(unfortunately) expected at this point and no reason for alarm.

\subsection{Quick Start}

For ease of installation and robustness, \ivoatex\ for now is designed
to be used from within a subdirectory of the directory containing the
document sources (rather than being installed globally).  Given that
it is fairly compact, having one copy per document seems acceptable.

So, the first step to use ivoatex is to create a development
directory:

\begin{lstlisting}[language=sh]
export DOCNAME=SampleDoc
# this would be your document's short name, e.g., RegTAP, SIAv2)
mkdir $DOCNAME
\end{lstlisting}

The DOCNAME -- which will turn up in URLs, standard identifiers, and the
like -- should be chosen to be both succinct and expressive, and it
should not contain non-alphanumeric characters (the examples given here
assume that, too).  A name
like \texttt{SimpleDALRegExt} probably marks the upper limit in terms of
length.

While it is clearly preferable if authors use IVOA's 
designated common version control
system -- at this point, this is
Volute\footnote{http://volute.googlecode.com} -- from the outset of
document development, it is possible to operate it locally as well.

\subsubsection{Installation from Archive (without Volute)}

Without version control, it is sufficient to obtain \ivoatex\ from a
distribution site and unpack it into the future document directory:

\begin{lstlisting}[language=sh]
cd $DOCNAME
curl http://soft.g-vo.org/ivoatex/ivoatex-latest.tar.gz \
  | tar -xvzf -k
\end{lstlisting}

\subsubsection{Installation with SVN version control}

While it is of course possible to keep \ivoatex\ in checkouts, too,
the recommended and more elegant way is to use \texttt{svn:externals}.

\begin{lstlisting}[language=sh]
export VOLUTEBASE="https://volute.googlecode.com/svn/trunk/projects"
export WG=?????
# this would be the Working Group name as represented in
# volute subdirectories: dal, dm, edu, grid, registry
svn import $DOCNAME $VOLUTEBASE/$WG/$DOCNAME
rmdir $DOCNAME
svn co $VOLUTEBASE/$WG/$DOCNAME $DOCNAME
cd $DOCNAME
svn propset svn:externals\
  "ivoatex $VOLUTEBASE/ivoapub/ivoatex" .
svn update
svn propset svn:ignore . --file ivoatex/svn-ignore.txt
\end{lstlisting}

This has the advantage that you get updates automatically and that it is 
trivial to feed back bibliography additions and patches to \ivoatex\
itself.

\subsubsection{Beginning the document}
\label{sect:beginning}

For convenience, the document production should start from some common
templates which are part of the \ivoatex\ distribution:

\begin{lstlisting}[language=sh]
cp ivoatex/Makefile.template Makefile
cp ivoatex/document.template $DOCNAME.tex
cp ivoatex/archdiag.png .
\end{lstlisting}

The next step is to fill out the makefile template.  
In \ivoatex\ version 0.4, this template looks like this:

\begin{lstlisting}[language=make]
# ivoatex Makefile.  The ivoatex/README for the targets available.

# short name of your document (e.g., RegTAP)
DOCNAME = ????

# count up; you probably do not want to bother with versions <1.0
DOCVERSION = 1.0

# Publication date, ISO format; update manually for "releases"
DOCDATE = ???

# What is it you're writing: NOTE, WD, PR, or REC
DOCTYPE = ???

# Source files for the TeX document (but the main file must always
# be called $(DOCNAME).tex
SOURCES = $(DOCNAME).tex

# List of pixel image files to be included in submitted package 
FIGURES = archdiag.png

# List of PDF figures (for vector graphics)
VECTORFIGURES = 

include ivoatex/Makefile
\end{lstlisting}

All lines with question marks need to be filled out.  The document date
is the publication date, which can be significantly different from the
current date.  After its initial setting, it should only be changed at
the time of submission to the document archive.  It is always in
DALI-style ISO format \citep{std:DALI}, e.g., 2014-03-31.

\texttt{SOURCES} is used in dependency processing.  It would be amended
when the source file is split into separate files or if material is
included into the document, e.g., via \texword{lstinputlisting}.
Graphics files included do not need to be given here.

\texttt{FIGURES} must contain the names of all bitmap graphics included
in the document; files missing here will be missing from the package for
distribution to the IVOA document repository, which will break the HTML
rendering.  As to \texttt{VECTORFIGURES}, see
sect.~\ref{sect:vectorgraphics}.

The template for the \TeX\ source contains several lines with
multiple question marks.  These must be filled out as well.

As illustrated in the template, both \texword{author} and
\texword{previousversion} support an optional argument giving an URL; for
\texword{author}, it should normally point to the respective person's
page in the IVOA wiki, for \texword{previousversion}, it should point to
the landing page of the respective document version in the IVOA document
repository.  Further automation for maintaining document history
certainly is desirable, and the authors welcome ideas for how this might
look like.

The architecture diagram specimen \texttt{archdiag.png} is only
necessary for documents on the recommendation track (i.e., notes in
general do not have one).  Architecture diagrams adapted to the concrete
document are currently prepared by the chair of the Technical
Coordination Group on request.  If no architecture diagram is present
within a document, the reference to it from \texttt{FIGURES} in the
makefile must be removed.

The document template assumes the source is written in UTF-8 encoding.
Authors are urged to keep lines shorter than 72 characters in input
files whenever possible in order to keep diffs useful and readable.
When edits are made, paragraphs should not normally be reflowed to avoid
large diffs for minor edits.

The PDF version of the document is built by the makefile's default rule,
so running \texttt{make} will usually by enough.  This will produce a
file \texttt{\$DOCNAME.pdf}.  If the makefile in the document directory
defines own rules, overriding the default target, this file name will
have to be given explicitly as the target name.

Other makefile targets for author use include:

\begin{itemize}
\item \texttt{biblio} updates the bibliography (i.e., runs \BibTeX);
running this is necessary after one of the bibliography files is updated
or when a new publication is referenced from the document.
\item \texttt{forecetex} rebuilds the PDF unconditionally (e.g., when TeX
asks to be rerun)
\item \texttt{\$DOCNAME.html} generates an HTML rendering of the document
\item \texttt{package} generates a zip file containing everything needed
for publication in the IVOA's document repository.   Obviously, 
\texttt{DOCVERSION}, \texttt{DOCTYPE}, and \texttt{DOCDATE} in the
Makefile should be updated as necessary before this target is built.
The result is a zip file with a name compliant to the IVOA document
standards \citep{std:docSTD}.
\end{itemize}

\begin{admonition}{Note}
Simply running \texttt{latex} or \texttt{pdflatex} directly
(rather than through make) is \emph{not} supported with \ivoatex.  Due
to the non-global installation of the support files, the \TeX\ run needs
a special environment that is prepared by the makefile.

Also note that bibliography processing must be initated manually by
running \texttt{make biblio}; unless authors checked in the bbl file
this produces, this must be run after a document checkout.
\end{admonition}

\subsection{Examples}

Examples for \ivoatex\ use include this 
document\footnote{https://volute.googlecode.com/svn/trunk/projects/ivoapub/ivoatexDoc}
or
RegTAP\footnote{https://volute.googlecode.com/svn/trunk/projects/registry/regtap}.
At the time of writing, about half a dozen IVOA documents within Volute
are based on \ivoatex.


\section{Authoring documents}
\label{sect:authoring}

While \ivoatex\ documents can be written much like any other \TeX\
document, it is advisable to follow certain standards and use special
facilities for common appearance, easier development, and possible
evolution of \ivoatex\ itself.

\subsection{\ivoatex\ Features}

\ivoatex\ provides a small set of macros and environments designed
to ease standards authoring.  These include:

\begin{bigdescription}
\item[\texword{author}, \texword{previousversion}] these are discussed
in sect.~\ref{sect:beginning}.
\item[\texword{xmlel}] a macro for marking up XML element or attribute
names and similar.  
\item[\texword{vorent}] for a name taken from VOResource, usually an
element or attribute name.
\item[\texword{admonition}] This is an environment for 
displayed boxes, intended for notes, tips, and the like.  
It takes an argument giving the head of the box, e.g.,

\begin{lstlisting}[language=TeX]
\begin{admonition}{Note}
Admonitions should not be overdone.  
Also, they are floating insertions.
\end{admonition}
\end{lstlisting}
\item[\texword{bigdescription}] This is an environment for definition
lists in the style of HTML \xmlel{dl}, and it will be translated into
one.  Use \verb|\item[term]| for the term to be defined.
\end{bigdescription}

The intention behind macros like \texword{xmlel} and \texword{vorent} is
that such terms are typeset uniformly across documents.  Further
semantic markup like this is planned for future releases, and document
authors are encouraged to contribute terms.

Also note that the title page is generated by the abstract environment.
Thus, all \ivoatex\ documents must have an abstract within the
\texword{abstract} environment.

\subsection{Listings, Verbatim Material}

\ivoatex\ documents should use the \texword{listings} package to include
source code snippets, XML fragments and the like.  It can be used after
a set of declarations like

\begin{lstlisting}[language=TeX]
\usepackage{listings}
\lstloadlanguages{XML,sh}
\lstset{flexiblecolumns=true,numberstyle=\small,numbers=left}
\end{lstlisting}

is included in the document preamble (i.e., before its
\verb|\begin{document}|). The languages preloaded should be adapted to
the document's needs. Numbering, of course, is optional.  Additional
languages supported that are likely relevant in a VO context include C,
fortran, python, SQL, and java, specified as above case-insensitively.

Actual listings are obtained with code like

\begin{verbatim}
\begin{lstlisting}[language=XML]
<example id="empty"/>
\end{lstlisting}
\end{verbatim}

Alternatively, entire files can be included like this:

\begin{verbatim}
\lstinputlisting[language=XML]{makeutypes.xslt}
\end{verbatim}

In the PDF rendering, the listings are pretty-printed.  In the HTML
rendering, the content is simply included in \xmlel{pre} elements right
now.

\subsection{References and Bibliography}

\ivoatex\ documents should use natbib and \BibTeX\ to manage references.
The package comes with a default bibliography in
\texttt{ivoatex/ivoabib.bib} containing records for many publications
likely to be cited by IVOA documents as well as (soon) all IVOA
documents already published either as recommendation or note. 

As usual in natbib, actual references are made through either
\verb|\citep{tag}|, yielding a form like ``(Einstein 1905)'',
or \verb|\citet{tag}|, yielding a form like ``Einstein (1905)''.
\ivoatex\ does not support variant forms of citep and citet (i.e., those
with optional arguments) yet; they will work in PDF output but fail in
HTML.  Contributions to improve this are welcome.


If a source does not already have a record in the bibliography that
comes with \ivoatex, authors are welcome to contribute new records into
its main repository.  Authors expecting that a reference really will not
be reused in other documents can have a local bibliography as well.
They will then need to amend the bibliography definition near the end of
the source file to, for instance,

\begin{lstlisting}[language=tex]
\bibliography{ivoatex/ivoabib,local}
\end{lstlisting}

(assuming the local records are stored in \texttt{local.bib}).

Changes that introduce new references or additions to the bibliography
require that \texttt{make biblio} is run afterwards.

\subsection{Graphics}

\ivoatex\ supports all bitmap graphics formats that pdflatex supports.
In practice, authors are encouraged to restrict themselves to JPEG, PNG,
and possibly GIF.  Currently, identical images are used for both PDF and
HTML renderings.  The recommended pattern for figures is

\begin{lstlisting}[language=tex]
\begin{figure}[thm]
\begin{center}
\includegraphics[width=0.9\textwidth]{mydiagram.png}
\end{center}
\caption{A diagram of what this is about.}
\label{fig:mydiag}
\end{figure}
\end{lstlisting}

-- this gives \LaTeX\ some leeway in placing the figure, defines the image
size in units of the page width, and centers the image itself.

All bitmap graphics in a document must  be listed in the makefile's
\texttt{FIGURES} variable.  If they are not, the HTML rendering will be
broken.

\subsubsection{Vector Graphics}
\label{sect:vectorgraphics}

The only vector graphics format supported in \ivoatex\ is PDF.  PDF
files can be directly used in \texword{includegraphics}.  The names of
such figures must be listed in the makefile's \texttt{VECTORFIGURES}
variable.

From \texttt{VECTORFIGURES}, \ivoatex\ arranges that, when a PDF figure 
\texttt{foo.pdf} is used, the HTML target depends on called
\texttt{foo.pdf.png}.  This PNG can be generated automatically by
\ivoatex\ using a combination of ghostscript and ImageMagick.  It may
sometimes be preferable to perform a custom conversion by hand (e.g.,
more compact representation with bilevel source images), in which case
the pre-rendered PNG should be included with version control.  This also
has the advantage that neither ghostscript nor ImageMagick are build
dependencies of the document.

\subsection{Tables}

In tables, rules should be used sparingly.  The standard pattern for tables is
something like

\begin{lstlisting}
\begin{table}[thm]
\begin{tabular}{p{0.35\textwidth}p{0.64\textwidth}}
\sptablerule
\textbf{Column Head}&\textbf{Another column head}\\
\sptablerule
A value & Another value\\
A value in row 2& And so on\\
\sptablerule
\caption{A sample table}
\label{table:extable}
\end{tabular}
\end{table}
\end{lstlisting}

The \texword{sptablerule} used here inserts a horizontal rule with some
extra spacing and will be rendered consistently in both PDF and HTML.
It should not, as a rule, used between table rows, it is intended
primarily to delimit the table itself as well as the the heading and the
body.

\subsection{Hyperlinks}

While \ivoatex\ puts no restrictions on the usage of hyperref features,
the preferred way to include links in \ivoatex\ documents is to use the
\texword{url} macro, i.e., use the URL itself as the anchor text.  In
this way, the link remains (to some extent) usable even if the document
is printed.  The alternative two-argument \texword{href} should
generally be avoided for that reason.  For instance,

\begin{lstlisting}
(this is bad:) The \href{http://ivoa.net}{IVOA} has issued
\href{http://ivoa.net/documents}{many standards}.
\end{lstlisting}

would severely degrade when printed and is hence discouraged, whereas

\begin{lstlisting}
The IVOA\footnote{\url{http://ivoa.net}} has issued many 
standards, all of which can be retrieved from 
\url{http://ivoa.net/documents}.
\end{lstlisting}

properly works on all of \ivoatex's target media.

\section{Customisation and Development}
\label{sect:impl}

This section discusses aspects of \ivoatex\ that are more technical in
nature.  Authors with a modicum of \TeX\ expertise are nevertheless
encouraged to read it.

\subsection{Technical Overview}

The central files in \ivoatex\ processing are

\begin{bigdescription}
\item[ivoa.cls] The class file, inheriting from \LaTeX's article class.
In it, several sections are delimited by comments like

\begin{lstlisting}
%%%%%%%%%%%%%%%%%%% Metadata definition macros
\end{lstlisting}

The file defines the markup rules for PDF processing, including
titlepage generation and extra macros and environments.  Its content is
ignored for HTML generation.

\item[tthdefs.tex] This file protects its contents from normal \TeX\
processing by a \verb|\iftth| conditional. This way, only tth sees
definitions made here. Each special feature defined in \texttt{ivoa.cls}
has a counterpart here, giving rules for translation to HTML.  This
usually encompasses emitting some HTML before and after the argument of
a TeX construct, where material between \verb|\begin{html}| and
\verb|\end{html}| is included literally in the HTML document.

\item[tth-ivoa.xslt] An XSLT stylesheet that postprocesses tth's output
and performs some operations that would be inconvenient to implement in
\texttt{tthdefs.tex}, in particular for the formatting of the opening
material.

\item[Makefile] This makefile is included by the user makefile in the
document directory proper.  It defines the rules given above as well as
some extra housekeeping rules like package building and building tth
from its source.

\end{bigdescription}

\subsection{Semantic Markup}

In order to make it support rich, semantic markup, \ivoatex\ needs to be
continuously developed.  In particular, it is good practice to define
macros for marking up values of certain datatypes, as with \ivoatex's
\texword{xmlel} and \texword{vorent}.  
Thus, whenever a document has multiple
instances of such values, authors should define macros and use these.
For instance, RegTAP deals with lots of concepts from its own
database schema and hence has

\begin{lstlisting}[language=TeX]
\definecolor{rtcolor}{rgb}{0.15,0.4,0.3}
\newcommand{\rtent}[1]{\texttt{\color{rtcolor} #1}}
\end{lstlisting}

in its document preamble to
define markup for ``RegTAP entity'', whereas
this note, as it mentions many words with a special meaning to \TeX, has

\begin{lstlisting}[language=TeX]
\definecolor{texcolor}{rgb}{0.4,0.1,0.1}
\newcommand{\texword}[1]{\texttt{\color{texcolor} #1}}
\end{lstlisting}

Such macros will be included in \ivoatex\ itself rather than an
individual document's preamble as they prove useful in multiple
documents.

\subsection{Custom Macros and Environments}

The tth translator used by \ivoatex\ ignores \texword{usepackage}.  Many
common packages are natively supported, but those that are not in
general need specific handling.  For instance, the \texword{lstlisting}
environment is not supported by tth, and hence in
\texttt{tthdefs.tex} there is

\begin{lstlisting}[language=TeX]
\newenvironment{lstlisting}[1][None]{%
    \begin{html}<pre class="listing">\end{html}
  }{%
    \begin{html}</pre>\end{html}
  }

\newcommand{\lstinputlisting}[2][None]{
  \verbatiminput{#2}
}
\end{lstlisting}

to have some sensible behaviour for the \ivoatex-supported constructs of
the package.

When a document requires special markup, it is likely that similarly
different implementations will be necessary for PDF and HTML output.
Using \texword{iftth} the implementations for the current output mode
can be selected (without the \texword{newif} mentioned in the tth
documentation, as that is already performed in \texttt{tthdefs.tex}.

For instance, RegTAP has many inline tables that need special spacing
the PDF rendering, whereas normal tables will do for them
in HTML.  It therefore
has in its preamble the definitions

\begin{lstlisting}
\iftth
  \newenvironment{inlinetable}{}{}
\else
  \newenvironment{inlinetable}{\vskip 1ex\vfil
    \penalty8000\vfilneg%
    \hbox to\hsize\bgroup\hss}
  {\hss\egroup\vspace{8pt}}
\fi
\end{lstlisting}


\subsection{Migration from Ivoadoc}

To ease migration from documents authored in ivoadoc, \ivoatex\ comes
with an XSLT stylesheet writing out a starting point for an ivoatex
version.  While many desirable features (e.g., extraction of titlepage
information) are not implemented and translated tables are incomplete,
the styleheet should still save time.  For XHTML-compliant ivoadoc
sources, the stylesheet is used like this:

\begin{lstlisting}[language=sh]
xsltproc ivoatex/fromivoadoc.xslt ivoadoc_source.html
\end{lstlisting}


\section{Desirable Features to be Implemented}

Frequently, certain parts of technical documents should be
machine-ge\-ne\-ra\-ted.  Examples of such parts include the automatic
extraction of XML schema snippets and their inclusion in the document
as provided by ivoadoc, or the TAP\_SCHEMA-derived tables of column
metadata in RegTAP.  As RegTAP shows, machine-generated content can
already be used within \ivoatex, but it would certainly be preferable if
some default framework for doing this were provided.

The inclusion of IVOA architecture diagrams as bitmaps prepared by the
TCG chair is unsatisfactory on both technical and organisational
grounds.  A common system allowing their easy generation based on SVG or
one of the \TeX-based graphics engines should be written.

A major drawback of \ivoatex's HTML output is that paragraphs are not actually
marked up as such.  Due to the \TeX\ processing model, their
reconstruction is non-trivial.  Hence in the generated HTML,
source-level paragraphs are rendered as text nodes separated by empty
HTML paragraph elements.  It would probably be possible to rectify this
in the XSLT postprocessing.

\appendix
\section{Changes from Previous Versions}

No previous versions yet.  
% these would be subsections "Changes from v. WD-..."
% Use itemize environments.


\bibliography{ivoatex/ivoabib}


\end{document}

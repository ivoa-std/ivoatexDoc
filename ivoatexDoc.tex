\documentclass[11pt,a4paper]{ivoa}
\input tthdefs

\usepackage{listings}
\lstloadlanguages{sh,make,[latex]tex}
\lstset{flexiblecolumns=true,numberstyle=\small,showstringspaces=False,
  identifierstyle=\texttt,defaultdialect=[latex]tex,language=tex}

\usepackage{todonotes}

\usepackage[utf8]{inputenc}

\definecolor{texcolor}{rgb}{0.4,0.1,0.1}
\newcommand{\texword}[1]{\texttt{\color{texcolor} #1}}

\iftth
  \newcommand{\BibTeX}{BibTeX}
\fi

\iftth
 \newcommand{\comicstuff}[1]{
    \begin{html}<span class="comic">#1</span>\end{html}}
\else
  \newcommand{\comicstuff}[1]{(HTML exclusive material)}
\fi
\customcss{custom.css}

\title{The \ivoatex\ Document Preparation System}

\ivoagroup{Standards and Processes}

\author[http://www.ivoa.net/cgi-bin/twiki/bin/view/IVOA/MarkusDemleitner]{Markus Demleitner}
\author[http://www.ivoa.net/cgi-bin/twiki/bin/view/IVOA/MarkTaylor]{Mark Taylor}
\author[http://www.ivoa.net/cgi-bin/twiki/bin/view/IVOA/PaulHarrison]{Paul Harrison}
\author[http://www.ivoa.net/cgi-bin/twiki/bin/view/IVOA/MarcoMolinaro]{Marco Molinaro}

\editor{Markus Demleitner}

\previousversion[http://www.ivoa.net/documents/ivoatexDoc/20160430]{Version
1.1}
\previousversion[http://www.ivoa.net/documents/ivoatexDoc/20150129]{Version
1.0}
       

\begin{document}

\SVN$Rev$
\SVN$Date$
\SVN$URL$

\begin{abstract}
This note describes the \ivoatex\ document preparation system for IVOA
standards and notes.  \ivoatex\ supports the production of
PDF and HTML renderings of the documents with sources in
plain text suitable for version control, as is desirable for normative
texts.  This note contains a user guide as well as a discussion of 
\ivoatex's dependencies and its implementation.  It refers to version
1.1 of the software.  While \ivoatex~is not usually explicitly
installed, it can be downloaded from \url{http://ivoatex.g-vo.org}.
\end{abstract}


\section*{Acknowledgments}

\ivoatex\ heavily draws from experiences made with previous markup-based
document preparation systems, in particular LaTeX classes and
infrastructure created by Sebasti\'en Derriere and Mark Taylor, as well
as Paul Harrison's XML-based ivoadoc system.

We thank tth's author, Ian Hutchinson, for generous technical support
and prompt provision of solutions in the upstream source where necessary.

\section{Introduction}

Creating and developing standards is a  big part of the operations of
the International Virtual Observatory Association (IVOA).
As these are normative texts, attention to detail is very important, and
being able to rigorously track changes to the documents is highly
advantageous.  

Standards are also often developed cooperatively, which means that
capabilities for branching and merging are desirable.  This strongly
suggests employing version control systems for document authoring.
Change tracking in software designed for editing office documents, to
the extent it is supported at all, usually requires significant
manual intervention, is optional, often used incorrectly, and frequently
lacks interoperability.  Led by these considerations, it was decided that
\ivoatex\ would have to be based on plain text source files.

As mandated by the IVOA Document Standards \citep{2017ivoa.spec.0517G},
finished documents have to be at
least available in PDF, while an additional HTML rendering for online
use is recommended.  A document preparation system should thus be
able  to produce documents in these formats in at least acceptable
quality.

With these constraints in mind, several possible solutions were
investigated.  Paul
Harrison's ivoadoc
system\footnote{\url{https://volute.g-vo.org/svn/trunk/projects/ivoapub/ivoadoc}} 
went for XHTML as
an input format and used XSLT2 and XML-FO as document processors.  While
this facilitated several interesting features  -- for instance,
automatic extraction and formatting of XML schema fragments or
straightvorward embedding of RDFa markup for machine-readable examples
--, it turned out that tooling issues were severe (e.g., reliable use
of SGML catalogs\footnote{This is important as retrieval of DTDs and
similar data from their commonly used system identifiers (i.e.,
typically W3C web servers) is at least undesirable and in practice
causes massive delays in formatting due to rate limiting on the part of
the W3C.}, non-free hyphenation patterns, classpath issues) and the use
of XML-FO for PDF generation yielded inferior renderings with little
prospect for improvements by third parties. Also, authors disliked
writing HTML tags.

Other options considered for source languages included docbook or one of
the newer lightweight markup languages (ReStructuredText, markdown,
etc).  In each case, there were concerns either regarding the system's power
and flexibility or its ease of installation and maintenance.

Meanwhile, several documents -- to mention just a few, SAMP, VOTable,
and VOUnits -- had successfully used \TeX-based systems typically
derived from work done in the early 2000s by Sebasti\'en Derriere.
\ivoatex\ essentially is a generalisation of these standards' formatting
systems, also inheriting from them the use of the \texttt{make} tool to
automate workflows.

\ivoatex\ was extended to relieve document editors from some of the
bookkeeping involved with producing IVOA standards and provide authors
with uniform solutions for common problems in standards typesetting.

In the remainder of this document, we give quick-start instructions
on installation and authoring in sect.~\ref{sect:quick} and continue
a more thorough discussion of \ivoatex's facilities with a special focus
on enabling proper automatic production of both HTML and PDF output in
sect.~\ref{sect:authoring}.  In sect.~\ref{sect:impl}, additional
details on the implemenation are given for the benefit of authors
planning to extend \ivoatex.  We close with a discussion of open issues
and desirable developments.

\section{Installation and Quick Start}
\label{sect:quick}

\subsection{Dependencies}

\ivoatex\ is designed to work more or less out of the box on common
POSIX-compliant systems; no non-free software is required for operation.
Its main uncommon dependency is the tth translator, which is a relatively 
compact program
written in highly portable C generated through lex used to translate LaTeX
to HTML.  As it is compact and portable, it is delivered with
\ivoatex~and built on demand.
Since \ivoatex's tth may at times offer
some enhancements over the upstream tth, using a system-installed tth is
discouraged.

The remaining dependencies include:

\begin{itemize}
\item A \LaTeX\ distribution with some commonly available packages (calc,
graphicx, xcolor, ifthen, doc, paralist, url, natbib, caption,
hyperref).
It is recommended to install TeXLive.
\item A sufficiently capable implementation of \texttt{make}, with GNU's
implementation recommended.
\item The XSLT1 processor \texttt{xsltproc} (a different processor can
be used, but that would probably require custom make rules).
\item The \texttt{gcc} compiler (another C compiler could be used; the
central makefile should probably be amended to allow easier changes
here).
\item The \texttt{zip} archiver for package generation.
\item \texttt{imageMagick} and \texttt{ghostscript} if vector graphics
is to be processed (which includes the architecture diagram).
\item \texttt{librsvg2} to let imageMagick process SVG.
\end{itemize}

On Windows, it is recommended to run \ivoatex\ within cygwin, where all
dependencies can easily be installed from cygwin's repository.  

On
Debian-derived systems, the dependencies should be present after
running a distribution-specific adaption of
\begin{lstlisting}[language=sh,basicstyle=\footnotesize]
apt-get install build-essential texlive-latex-extra zip xsltproc\
  texlive-bibtex-extra imagemagick ghostscript cm-super librsvg2-bin
\end{lstlisting}
(cm-super contains vector versions of computer modern fonts in T1
encoding), on RPM-based systems something like
\begin{lstlisting}[language=sh]
yum install texlive-scheme-full libxslt make gcc zip\
  ImageMagick ghostscript
\end{lstlisting}
should pull in everything that is necessary.

With OS X, a convenient way to obtain the dependencies is to install
MacPorts\footnote{\url{https://www.macports.org/}} and then run
\begin{lstlisting}[language=sh]
port install ImageMagick  libxslt ghostscript +full
\end{lstlisting}
The canonical OS\,X \TeX\ distribution is the Mac\TeX\ version of
\TeX Live\footnote{\url{https://www.tug.org/mactex/}}.  It is also
possible to build \TeX\ using MacPorts (with \texttt{port install
  texlive}), but this may result in a slightly non-standard
distribution.\footnote{See
  \url{http://tex.stackexchange.com/questions/97183/}.
  Also, MacPorts does add
  \texttt{texlive} as a dependency on many packages, and so frequently
  \emph{insists} on trying to build it; if you want to prevent this,
  there is some discussion at
  \url{http://comments.gmane.org/gmane.os.apple.macports.user/21526}.}

To see if the prerequisites are there and compatible with \ivoatex, try
building an updated version of this document from its volute source:
\begin{lstlisting}[language=sh,basicstyle=\footnotesize]
svn co https://volute.g-vo.org/svn/trunk/projects/ivoapub/ivoatexDoc
cd ivoatexDoc
make biblio   # update the bibliography
make forcetex # make a PDF ignoring timestamps
make ivoatexDoc.html  # make an html document
make package # make a zipfile for IVOA submission
\end{lstlisting}
During HTML generation, various diagnostics both 
from tth and from xsltproc (unknown commands,
unexpected end tags, and the like) are
expected at this point and no reason for alarm; we work on reducing the
amount of spurious error messages.

\subsection{Basic \ivoatex~operation}

For ease of installation and robustness, \ivoatex\ for now is designed
to be used from within a subdirectory of the directory containing the
document sources (rather than being installed globally).  Given that
it is fairly compact, having one copy per document seems acceptable.

So, the first step to use \ivoatex\ is to create a development
directory:

\begin{lstlisting}[language=sh]
export DOCNAME=SampleDoc
# this would be your document's short name, e.g., RegTAP, SIAv2)
mkdir $DOCNAME
\end{lstlisting}

The DOCNAME -- which will turn up in URLs, standard identifiers, and the
like -- should be chosen to be both succinct and expressive, and it
should not contain non-alphanumeric characters (the examples given here
assume that, too).  A name
like \texttt{SimpleDALRegExt} probably marks the upper limit in terms of
length.

While it is clearly preferable if authors use IVOA's 
designated common version control
system -- at this point, this is
Volute\footnote{\url{http://volute.g-vo.org}} -- from the outset of
document development, it is possible to operate it locally as well.

\subsubsection{Installation from Archive (without Volute)}

Without version control, it is sufficient to obtain \ivoatex\ from a
distribution site and unpack it into the future document directory:

\begin{lstlisting}[language=sh]
cd $DOCNAME
curl http://ivoatex.g-vo.org/ivoatex-1.1.tar.gz \
  | tar -xvzf -k
\end{lstlisting}

\subsubsection{Installation with SVN version control}

While it is of course possible to keep \ivoatex\ in checkouts, too,
the recommended and more elegant way is to use \texttt{svn:externals}.

\begin{lstlisting}[language=sh,basicstyle=\footnotesize]
export VOLUTEBASE="https://volute.g-vo.org/svn/trunk/projects"
export WG=?????
# this would be the Working Group name as represented in
# volute subdirectories: dal, dm, edu, grid, registry
svn import $DOCNAME $VOLUTEBASE/$WG/$DOCNAME
rm -r $DOCNAME
svn co $VOLUTEBASE/$WG/$DOCNAME $DOCNAME
cd $DOCNAME
svn propset svn:externals\
  "ivoatex $VOLUTEBASE/ivoapub/ivoatex" .
svn update
svn propset svn:ignore . --file ivoatex/svn-ignore.txt
\end{lstlisting}

This has the advantage that \ivoatex~updates are automatically pulled in
and that it is
trivial to feed back bibliography additions and patches to \ivoatex\
itself.

\subsubsection{Beginning the document}
\label{sect:beginning}

\paragraph{Main metadata in the Makefile}

For convenience, the document production should start from some common
templates which are part of the \ivoatex\ distribution:

\begin{lstlisting}[language=sh]
cp ivoatex/Makefile.template Makefile
cp ivoatex/document.template $DOCNAME.tex
# Notes don't need the following
cp ivoatex/archdiag-full.xml role_diagram.xml 
\end{lstlisting}

The next step is to fill out the makefile template.  
As of the formatting of this document, this template looks like this:

\lstinputlisting[language=make,basicstyle=\footnotesize]{ivoatex/Makefile.template}

All lines with question marks must be filled out.  The document date
is the publication date, which can be significantly different from the
current date.  After its initial setting, it should only be changed at
the time of submission to the document archive.  It is always in
DALI-style ISO format \citep{2017ivoa.spec.0517D}, e.g., 2014-03-31.

\texttt{SOURCES} is used in dependency processing.  It would be amended
when the source file is split into separate files or if material is
included into the document, e.g., via \texword{lstinputlisting}.
Graphics files included do not need to be given here, as the document
will automatically depend on them.

\texttt{FIGURES} must contain the names of all bitmap graphics included
in the document; files missing here will be missing from the package for
distribution to the IVOA document repository, which will break the HTML
rendering.  As to \texttt{VECTORFIGURES}, see
sect.~\ref{sect:vectorgraphics}.

\texttt{AUX\_FILES} is intended for files that should be included in the
upload to the IVOA document repository while not taking part in the
actual formatting.  This in particular concerns XML Schema files, for
which the IVOA maintains a separate repository, but is by no means
limited to them.

\paragraph{Additional metadata in the \LaTeX~source}

The template for the \TeX\ source contains several lines with
multiple question marks.  These must be filled out as well.

As illustrated in the template, both \texword{author} and
\texword{previousversion} support an optional argument giving an URL; for
\texword{author}, it should normally point to the respective person's
page in the IVOA wiki, for \texword{previousversion}, it should point to
the landing page of the respective document version in the IVOA document
repository.  Further automation for maintaining document history
certainly is desirable, and the authors welcome ideas for how this might
look like.

\texword{ivoagroup} should contain the name of the (one) IVOA working or
interest group under which auspices the document development mainly
happens.  Pick one from the following list\footnote{We do not
enforce the use of a controlled vocabulary here; automatic submission
with \texttt{make upload} will fail if you invent something here,
though.}:

\begin{compactitem}
\item Applications
\item DAL
\item Data Access Layer
\item Data Models
\item Grid and Web Services
\item Registry
\item Data Curation and Preservation
\item Standards and Processes
\item Semantics
\item Theory
\item VO Event
\item Time Domain
\item Education
\item No Group (only possible for Notes)
\end{compactitem}


\paragraph{The Archtecture Diagram}

An architecture diagram is only 
necessary for documents on the recommendation track.  Notes may have
one, but since Notes are not shown on it, that is rather
unusual.  If you do not want to include an architecture diagram, remove
\texttt{role\_diagram.svg} from \texttt{FIGURES} and
\texttt{role\_diagram.pdf} from \texttt{SOURCES} in the makefile.

To prepare an architecture diagram, copy
\texttt{ivoatex/archdiag-full.xml} to \texttt{role\_diagram.xml}.  Then edit
\texttt{role\_diagram.xml} and remove all references to standards unrelated
to the current document.  As a rule, all \texttt{rec} elements for
standards not mentioned in
``Role within the IVOA Architecture'' should be removed.  Finally, use a
\xmlel{thisrec} element for the current standard.  An architecture
diagram for VOResource would thus be specified like this:

\begin{lstlisting}[language=XML,basicstyle=\footnotesize]
<archdiag xmlns="http://ivoa.net/archdiag">
  <thisrec name="VOResource" x="55" y="155"/>
  <rec name="RegTAP" x="55" y="180"/>
  <rec name="RegistryInterface" x="55" y="205"/>
  <!-- and a few more -->
</archdiag>
\end{lstlisting}

The standard-specific
architecture diagram will be built by saying \texttt{make
role\_diagram.svg} and can be viewed using, for instance, common web
browsers.  Please refrain from editing the SVG with vector graphics
programs.  In the PDF renderings of the document, 
a PDF version of the architecture diagram is
required.  This is built using \texttt{make role\_diagram.pdf}, and a
stand-in will be built if your system lacks the software to do this
conversion.  It should be fine to use common vector graphic tools like
inkscape to perform the conversion in that case.
As long as the tooling situation regarding \LaTeX~and SVG is as
unsatisfactory as it is, please commit both \texttt{role\_diagram.svg} and
\texttt{role\_diagram.pdf} to the version control system.

\paragraph{Source code conventions}

\ivoatex\ requires the source to be written in UTF-8 encoding, since the
references shipped with it are in UTF-8.
Authors are urged to keep lines shorter than 72 characters in input
files whenever possible in order to keep diffs useful and readable.
When edits are made, paragraphs should not normally be reflowed to avoid
large diffs for minor edits.  Authors desiring a reflow after many edits
are encouraged to concentrate them in a separate, reflow-only commit.

\paragraph{Makefile targets}

The PDF version of the document is built by the makefile's default rule,
so running \texttt{make} will usually by enough.  This will produce a
file \texttt{\$DOCNAME.pdf}.

Other makefile targets for author use include:

\begin{itemize}
\item \texttt{biblio} updates the bibliography (i.e., runs \BibTeX);
running this is necessary after one of the bibliography files is updated
or when a new publication is referenced from the document.
\item \texttt{forecetex} rebuilds the PDF unconditionally (e.g., when TeX
asks to be rerun)
\item \texttt{\$DOCNAME.html} generates an HTML rendering of the
document; at this point, this will typically emit quite a bit of
spurious diagnostics.  Unfortunately, real problems may hide within.  
Therefore, we currently recommend a visual inspection of the resulting
HTML before submission.
\item \texttt{package} generates a zip file containing everything needed
for publication in the IVOA's document repository.   Obviously, 
\texttt{DOCVERSION}, \texttt{DOCTYPE}, and \texttt{DOCDATE} in the
Makefile should be updated as necessary before this target is built.
The result is a zip file with a name compliant to the IVOA document
standards \citep{2017ivoa.spec.0517D}.
\item \texttt{arxiv-upload}  produces a file arxiv-upload.tar.gz for
the current document.  ArXiv uploads are automatically done by the
document coordinator for RECs and ENs, and editors \emph{should not} do
them themselves.  Authors of Notes may upload their documents to arXiv.
\item \texttt{upload} produces a package, shows metadata to be uploaded to the
IVOA document repository (make sure everything is as you expect it) and
then uploads the package.
\item \texttt{generate} is used to update machine-generated content in
the document, typically by the main editor.  See
section~\ref{sect:generated} for details.
\end{itemize}

\begin{admonition}{Note}
Simply running \texttt{latex} or \texttt{pdflatex} directly
(rather than through make) is \emph{not} supported with \ivoatex.  Due
to the non-global installation of the support files, the \TeX\ run needs
a special environment that is prepared by the makefile.

Also note that bibliography processing must be initated manually by
running \texttt{make biblio}; unless authors checked in the bbl file
this produces, this must be run after a document checkout.
\end{admonition}

\subsection{Examples}

Examples for \ivoatex\ use include this 
document\footnote{https://volute.g-vo.org/svn/trunk/projects/ivoapub/ivoatexDoc}
or
RegTAP\footnote{https://volute.g-vo.org/svn/trunk/projects/registry/regtap}.
At the time of writing, numerous other documents based on \ivoatex~can
be found on Volute.


\section{Authoring documents}
\label{sect:authoring}

While \ivoatex\ documents can be written much like any other \TeX\
document, it is advisable to follow certain standards and use special
facilities for common appearance, easier development, and possible
evolution of \ivoatex\ itself.

\subsection{\ivoatex\ Features}

\ivoatex\ provides a small set of macros and environments designed
to ease standards authoring.  These include:

\begin{bigdescription}
\item[\texword{author}, \texword{previousversion}, \texword{ivoagroup}] these are discussed
in sect.~\ref{sect:beginning}.
\item[\texword{xmlel}] a macro for marking up XML element or attribute
names and similar.  
\item[\texword{vorent}] for a name taken from VOResource or its
extensions, usually an
element or attribute name.
\item[\texword{admonition}] This is an environment for 
displayed boxes, intended for notes, tips, and the like.  
It takes an argument giving the head of the box, e.g.,

\begin{lstlisting}[language=TeX]
\begin{admonition}{Note}
Admonitions should not be overdone.  
Also, they are floating insertions.
\end{admonition}
\end{lstlisting}
\item[\texword{bigdescription}] This is an environment for definition
lists in the style of HTML \xmlel{dl}, and it will be translated into
one.  Use \verb|\item[term]| for the term to be defined
(the construct this item is in is a \texword{bigdescription}).
\item[\texword{auxilaryurl}] A macro expanding a URL relative to the
current document to a full URL.  This is discussed more extensively in
sect.~\ref{sect:links}
\item[\texword{inlinetable}] This is an environment for showing tables
in-place (rather than as an insertion potentially somewhere else, as
LaTeX's table environment does).  This may lead to rather empty pages
when the tables get longer.
\end{bigdescription}

The intention behind macros like \texword{xmlel} and \texword{vorent} is
that such terms are typeset uniformly across documents.  Further
semantic markup like this is planned for future releases, and document
authors are encouraged to contribute terms.

Also note that the title page is generated by the abstract environment.
Thus, all \ivoatex\ documents must have an abstract within the
\texword{abstract} environment.

\subsection{Listings, Verbatim Material}

\ivoatex\ documents should use the \texword{listings} package to include
source code snippets, XML fragments and the like.  While the ivoa class
requires the package as of version 1.1 (and hence you can use the
\texword{lstlisting} environment without an explicit
\texword{usepackage} declaration), you will usually want to configure
the package in the document preamble (i.e., before its
\verb|\begin{document}|), perhaps like this:

\begin{lstlisting}[language=TeX]
\lstloadlanguages{XML,sh}
\lstset{flexiblecolumns=true,tagstyle=\ttfamily,
  showstringspaces=False}
\end{lstlisting}

Additional
languages supported that are likely relevant in a VO context include C,
fortran, python, SQL, and java, specified as above case-insensitively.

The setup in the example (flexible columns, tags in
typewriter, no explicit blanks even in strings) 
produces reasonably pleasant output for a wide range of languages.

Actual listings are obtained with code like
\begin{verbatim}
\begin{lstlisting}[language=XML]
<example id="empty"/>
\end{lstlisting}
\end{verbatim}
Alternatively, entire files can be included like this:
\begin{verbatim}
\lstinputlisting[language=XSLT]{makeutypes.xslt}
\end{verbatim}
In the PDF rendering, the listings are pretty-printed.  In the HTML
rendering, the content is, currently, simply included in \xmlel{pre} elements.

If a more compact rendering of listings is desired, for instance,
because larger portions of source code are required in the document,
listings' \texword{basicstyle} option should be used together with one
of LaTeX standard size macros.  This could be in the argument of
\texword{lstset} in the preamble for a global setting, or on a
case-by-case basis as in

\lstinputlisting[language=tex]{verbatimstyles.tex}

As of version 1.1 of \ivoatex, only \texword{footnotesize} is actually
formatted by the CSS embedded in the HTML document, and we believe
listings should not be much smaller than that anyway.  As the options
are translated into CSS classes, it is fairly easy to add further
formatting functionality on a document-by-document basis, though.

\subsection{References and Bibliography}

\subsubsection{Built-in Bibliographies}
\ivoatex\ documents should use natbib and \BibTeX\ to manage references.
The package comes with two default bibliographies:

\begin{itemize}
\item \texttt{ivoatex/ivoabib.bib} containing records for many publications
likely to be cited by IVOA documents.  It also contains ``historical''
records for IVOA standards, the \BibTeX\ tags of which start with
\texttt{std:}.  Do not use these any more in new documents, they are
only present in order to not break legacy documents.
\item\texttt{ivoatex/docrepo.bib} containing records for IVOA documents
listed in ADS.  At this writing, this concerns all recommendations.  A
selection of Notes will be added at a later point.
\end{itemize}

Changes to the document that introduce new references or changes to the
bibliography require that \texttt{make biblio} is run before changes
become visible.

\ivoatex\ comes with a bibliography style of its own, derived from
\texttt{agsm.bst}.  The custom bibliography style was derived to
optimise some types of sources uncommon outside of the VO community, in
particular IVOA recommendations and notes.  Users are welcome to improve
\texttt{ivoatex/ivoa.bst}.

\subsubsection{Citation Style}

As usual in natbib, actual references are made through either writing
\verb|\citep{tag}|, yielding a form like ``(Einstein 1905)'',
or \verb|\citet{tag}|, yielding a form like ``Einstein (1905)''.
\ivoatex\ does not support variant forms of citep and citet (i.e., those
with optional arguments) yet; they will work in PDF output but fail in
HTML.  Contributions to improve this are welcome.

To reference an IVOA recommendation, locate its bibcode either using ADS
or directly within \texttt{ivoatex/docrepo.bib} and then use one of the
cite macros.  The preferred style is to introduce a short name for the
standard once with a citation and then use that short name in the
remainder of the document to have expressive texts not overciting. For
instance,

\begin{lstlisting}[language=tex]
IVOA Identifiers \citep{2016ivoa.spec.0523D} introduces URIs to
reference Registry records, which are typically transmitted in
VOResource \citep{2018ivoa.spec.0625P} format.  Both VOResource 
and IVOA Identifiers are based on various W3C standards.
\end{lstlisting}

will come out as

\begin{quotation}
IVOA Identifiers \citep{2016ivoa.spec.0523D} introduces URIs to
reference Registry records, which are typically transmitted in
VOResource \citep{2018ivoa.spec.0625P} format.  Both VOResource and IVOA
Identifiers are based on W3C standards.
\end{quotation}

Recommendations should always be cited in the last version available
at the time of writing\footnote{If a published recommendation is missing
in the bibliography, this can be fixed by cd-ing into the ivoatex folder
and saying \texttt{make docrepo.bib}.}.  If works in progress (working
drafts, proposed recommendations are to be cited, authors should create
a local bibliography (see below).

If a non-temporary source (like a journal article or a software
programme) does not already have a record in \texttt{ivoabib.bib},
authors are welcome to contribute new records there.

\subsubsection{Local Bibliographies}

When a reference is really only relevant to a single document or is
conceptually non-permanent -- this, in particular, pertains to Working
Drafts or Proposed Recommendations to be cited -- should be kept in a
local bibliography.  To enable such a document-local bibliography,
the bibliography declaration at the foot of the document needs to be
changed to

\begin{lstlisting}[language=tex]
\bibliography{ivoatex/ivoabib,ivoatex/docrepo,localrefs}
\end{lstlisting}

Then a file \texttt{localrefs.bib} is created and checked into the
version control repository.  An example \BibTeX\ record for an
in-progress document is

\begin{lstlisting}
@Misc{wd:Datalink,
  author={Patrick Dowler and Francois Bonnarel and 
    Laurent Michel and Tom Donaldson and David Languignon},
  editor={Patrick Dowler},
  howpublished={{IVOA Working Draft 22 October 2013}},
  title={DataLink},
  year=2013,
  url={http://ivoa.net/documents/DataLink/20131022/}
}
\end{lstlisting}

Note that \emph{howpublished} contains the precise document date and the
\emph{url} points to the actual versioned landing page, not the generic
one for the standard.

\subsection{Graphics}

\subsubsection{Bitmap Graphics}

\ivoatex\ supports all bitmap graphics formats that pdflatex supports.
In practice, authors are encouraged to restrict themselves to JPEG, PNG,
and possibly GIF.  Currently, identical images are used for both PDF and
HTML renderings.  The recommended pattern for figures is
\begin{lstlisting}[language=tex]
\begin{figure}[th]
\begin{center}
\includegraphics[width=0.9\textwidth]{mydiagram.png}
\end{center}
\caption{A diagram of what this is about.}
\label{fig:mydiag}
\end{figure}
\end{lstlisting}
This gives \LaTeX\ some leeway in placing the figure, defines the image
size in units of the page width, and centers the image itself.

All bitmap graphics in a document must  be listed in the makefile's
\texttt{FIGURES} variable.  If they are not, the HTML rendering will be
broken.

\subsubsection{Vector Graphics}
\label{sect:vectorgraphics}

The only vector graphics format supported in \ivoatex\ is PDF.  PDF
files can be directly used in \texword{includegraphics}.  The names of
such figures must be listed in the makefile's \texttt{VECTORFIGURES}
variable.

From \texttt{VECTORFIGURES}, \ivoatex\ arranges that, when a PDF figure 
\texttt{foo.pdf} is used, the HTML target depends on a file called
\texttt{foo.png}.  This PNG can be generated automatically by
\ivoatex\ using a combination of ghostscript and ImageMagick.  It may
sometimes be preferable to perform a custom conversion by hand (e.g.,
more compact representation with bilevel source images), in which case
the pre-rendered PNG should be included in the version controled
repository.  This also
has the advantage that neither ghostscript nor ImageMagick are build
dependencies of the document.

\subsection{Tables}

In tables, rules should be used sparingly.  The standard pattern for tables is
something like
\begin{lstlisting}
\begin{table}[th]
\begin{tabular}{p{0.35\textwidth}p{0.64\textwidth}}
\sptablerule
\textbf{Column Head}&\textbf{Another column head}\\
\sptablerule
A value & Another value\\
A value in row 2& And so on\\
\sptablerule
\caption{A sample table}
\label{table:extable}
\end{tabular}
\end{table}
\end{lstlisting}

The \texword{sptablerule} used here inserts a horizontal rule with some
extra spacing and will be rendered consistently in both PDF and HTML.
It should not, as a rule, used between table rows. It is intended
primarily to delimit the table itself as well as the the heading and the
body.

\subsection{Hyperlinks}
\label{sect:links}

While \ivoatex\ puts no restrictions on the usage of hyperref features,
the preferred way to include links in \ivoatex\ documents is to use the
\texword{url} macro, i.e., use the URL itself as the anchor text.  In
this way, the link remains (to some extent) usable even if the document
is printed.  The alternative two-argument \texword{href} should
generally be avoided as it fails on paper.  For instance,
\begin{lstlisting}
(this is bad:) The \href{http://ivoa.net}{IVOA} has issued
\href{http://ivoa.net/documents}{many standards}.
\end{lstlisting}
would severely degrade when printed and is hence discouraged, whereas
\begin{lstlisting}
The IVOA\footnote{\url{http://ivoa.net}} has issued many 
standards, all of which can be retrieved from 
\url{http://ivoa.net/documents}.
\end{lstlisting}
works properly on all of \ivoatex's target media.

With non-HTTP URIs, it is recommended to use hyperref's
\texword{nolinkurl} macro (rather than an unadorned \texword{texttt} or
similar); advantages include that line breaking is better with
nolinkurl, less manual escaping is necessary, and, if desired, 
such URIs can be more easily styled.  An example:

\begin{lstlisting}
The value of the \xmlel{standardID} attribute will be
be \nolinkurl{ivo://ivoa.net/std/Registry#OAI-2.0}.
\end{lstlisting}

A special feature related to links is the \texword{auxiliaryurl} macro.
It is used when a document references a resource coming with the
document and versioned with it, but not included with the document text.
Examples for such material could be helper scripts, simple validators,
larger bits of sample data, and the like.  Note that vocabularies and
XML schema files should not be distributed in this way, as they have
conventional places within the ivoa.net URL hierarchy.

For instance, \verb|\auxiliaryurl{custom.css}| expands to
$$\hbox{\auxiliaryurl{custom.css},}$$ in this document, 
which is, for each release of this
document, the URL the example custom CSS file mentioned in the example
in sect.~\ref{sect:customcss} is found at.  The preferred way to include
such (long) URLs is in footnotes, as they typcially cannot be hyphenated.

In order to make \ivoatex~include the file linked to in this way in the
package submitted to the document repository, you must add the argument
of \texword{auxiliaryurl} to \verb|AUX_FILES| in the Makefile.

\subsection{Editorial tools}

When authoring standards, it is sometimes necessary to include
editorial comments of the type ``Need to clarify'' or ``Specification
incomplete''.  We recommend to use the todonotes package for such
pieces of text\footnote{Full documentation is available at
\url{http://www.tex.ac.uk/CTAN/macros/latex/contrib/todonotes/todonotes.pdf}.}.
The recommended usage is like
\begin{lstlisting}
...
\usepackage{todonotes}
...
\begin{document}
\todo{This is an example for a editorial note}.
\end{lstlisting}

A rendering of such a to-do note is shown in this paragraph
\todo{This is an example for a editorial note}. In HTML output,
only the simple \texword{todo} macro without options is supported, and
text is simply displayed inline right now.  Note in particular that
todonote's \texword{obeyDraft} and \texword{obeyFinal} package options
are ignored.  Although \ivoatex\ does not enforce it (yet), finished
recommendations should have no todo items in them.

While todonotes is a useful tool for standards development, we
discourage the use of packages to mark up changes, as maintaining such
markup usually is very hard, and version control offers a more
manageable solution to the problem such packages attempt to solve.

\subsection{Version Control System Information}
\label{sect:vcs}

It is recommended to include basic metadata obtained from the version
control system into \ivoatex~Documents where available.  Basic support
for Volute (i.e., subversion) is built into \ivoatex.  It is based on
subversion's
keyword substitution, which therefore needs to be enabled on the 
main document source file.
While any keys might be used, \ivoatex's default support deals with:

\begin{lstlisting}[language=sh]
svn propset svn:keywords "Date Rev URL" $DOCNAME.tex
\end{lstlisting}

(only Rev is mandatory).
In the document itself, the text for subversion's replacement needs to
be included.  Inititally, this must look like this:

\lstinputlisting[language=tex]{svnsubstitution.tex}

Subversion will, at every commit, enter the current values in a
controlled fashion, which will then be picked up by \ivoatex's macros.
Due to the way this information is processed by \TeX, after adding the above
text, a commit must be made before the document can be built again.

For non-English locales, subversion's substitution poses the problem that
localized information will be entered into the document, which is
unwelcome for many resaons (non-ASCII characters, ambiguous date
formats).  We therefore recommend to run SVN in the C
locale, at least while working on IVOA documents.  Users affected with
on-C locales
could in a Bourne-like shell use a construct like
\begin{lstlisting}[language=sh]
alias svn='LC_ALL=C svn'
\end{lstlisting}
and forget about the problem.


\subsection{Generated Content}
\label{sect:generated}

Sometimes it is desirable to have parts of a document generated through
some sort of ivoatex-external process.  Examples include copying
documentation from XML Schema files into the standard document 
or obtaining column metadata for standard
data models from a live TAP\_SCHEMA.

For such cases, \ivoatex\ offers a python script
\texttt{update\_generated.py}, which is executed by the
\texttt{generate} make target.  It simply looks for structured comments
in the main document and replaces what is between them with generated
contents.  The opening line consists of a \TeX~comment introducer, a
blank, the literal \texttt{GENERATED:}, another blank, and a command
line.  The closing line is a comment consisting of \texttt{/GENERATED}.

On running \texttt{make generate}, the material between the opening and
the closing line is replaced by the output of the command.

For instance, in the following snippet the material between the comments
was inserted by \texttt{make generate}:

\begin{lstlisting}[language=TeX]
% GENERATED: echo This is generated content
This is generated content

% /GENERATED
\end{lstlisting}

\texttt{make generate} will not replace any content in the document source
if even just one command fails to execute as indicated by the
command's return code; it is transactional in this sense.

In addition to allowing arbitrary shell
commands, \texttt{update\_generated.py} has a facility that allows
calling special python functions from within documents: Commands
starting with an exclamation mark (``!'') are translated to calls to
appropriately named python functions defined within
\texttt{update\_generated.py}.

Currently, there are two such builtin commands; both are somewhat
experimental features that may change when more experience has been
gathered as to their usefulness.

One command is \texttt{taptable}, which extracts documentation from a
live \texttt{TAP\_SCHEMA}\footnote{It is in use in the
\href{https://volute.g-vo.org/svn/trunk/projects/registry/discovercollections}{Discovering
Data Collections Within Services} IVOA Note.}.  The access URL of the
live TAP service must be specified in an environment variable TAPURL in
the Makefile, for instance

\begin{lstlisting}[language=sh]
export TAPURL=http://dc.g-vo.org/tap
\end{lstlisting}

Then, in the document one can use the structured comment

\begin{lstlisting}[language=TeX]
% GENERATED: !taptable tap_schema.tables
% /GENERATED
\end{lstlisting}

After a \texttt{make generate}, the LaTeX source for a table describing
the columns of \texttt{tap\_schema.tables} will be between the two
comment lines; re-running \texttt{make generate} will replace that
content with a refreshed version.

The second command implemented is \texttt{schemadoc} that formats
documentation for a type in an XML schema file.  This only works
properly if the XML schema defines a \xmlel{vm:targetPrefix} element in
the way pioneered by Ray Plante in
VOResource.\footnote{See
\url{http://volute.g-vo.org/svn/trunk/projects/registry/VOResource} for
an example of a schema file instrumented for schemadoc.}
With an appropriately instrumented schema, a document can say

\begin{lstlisting}[language=TeX]
% GENERATED: !schemadoc SchemaSource.xsd MyType
% /GENERATED
\end{lstlisting}

to produce documentation for the schema type \xmlel{MyType} within
the XML schema file \texttt{SchemaSource.xsd}.

\ivoatex~will never execute \texttt{update\_generated.py} as part of a
dependency chain; it is intended that \texttt{make generate} must always
been manually triggered.  On the one hand, this is because its
dependencies cannot be generically modelled, given that arbitrary
commands can be executed.  Document authors are also discouraged from
providing such dependency information -- it is fairly common that
content generation depends on the availability of external resources
(e.g., databases or network services), and a document build should not
fail just because these are unavailable.

We mention in passing that generated content puts potentially executable
material into documents, which is of course an attack vector for
malicious software.  However, calling \texttt{make update} is no
additional security risk, as presumably the author of the makefile is
identical to the document author, and the makefile can already contain
arbitrary commands that would be executed on the calling user's behalf.

\subsection{The Standards Record}

IVOA recommendations and endorsed notes must be registered using the
schema from StandardsRegExt \citep{2012ivoa.spec.0508H}.  A major use
case for this is to allow references to standards from other registry
records. The prototypcial example for this is the \xmlel{standardID}
attribute of VOResource \xmlel{capability} elements that contains
identifiers (and possibly standard keys as fragment identifiers)
declaring ``this service works as specificed by that standard''.

Registration of IVOA-approved standards happens through the registry of
registries (RofR, authority ivoa.net).  The preparation of the registry
records, however, is up to the document editor.  It is strongly
recommended to keep and maintain the registry record in version control
alongside with the document source.

To do that, create a standards record from ivoatex's template (assuming
operation within an svn checkout):

\begin{lstlisting}[language=sh]
$ svn copy ivoatex/stdrec-template.xml stdrec.xml
\end{lstlisting}

In the resulting file, the items one has to change are marked with four
hash marks; the elements \xmlel{schema} and \xmlel{key} can be removed
or repeated as required by a particular standard, all other elements
present should be given at least once.  Please remove the explanatory
comments as you go. 

A sample record filled out for this document is kept in the subversion
repository (but not uploaded into the Registry, as Notes do not usually
require registration).

Most of the metadata to be filled in actually already is available in
structured form in \ivoatex.  Contributions that help automating
creation and maintenance of stdrec.xml exploiting this are welcome.

With the upload of a proposed recommendation to the document repository,
the editor should send the standards record to the registry of
registries by e-mail\footnote{See \url{http://rofr.ivoa.net/} for
contact information.} and then re-submit the record with each upload.

\subsection{Submitting a Document}

Snapshots of documents should be regularly submitted to the document
repository as part of the IVOA review process.  While \ivoatex~could
still make this a bit more streamlined, there is some automation for
this process present.  Consider the following recommendation for the
release process (of course, replace \verb|${DOCNAME}| with the DOCNAME
as defined in the Makefile):

\begin{enumerate}
\item Add the version-specific URL and designation (e.g.,
``WD-20100801'') using \texword{previousversion} of the currently
published document (i.e., the version you are replacing) in the document
header.
\item Run \verb|make| and inspect \verb|${DOCNAME}.log|.  There should
obviously be no undefined citations or internal references.  Also, try
to avoid overfull hboxes (underfull hboxes are not always avoidable).
\item Have another round of proofreading on \verb|${DOCNAME}.pdf|.  Also
critically inspect the architecture diagram and the relationships to
other standards, in particular whether new versions of those have
reached REC in the meantime.
\item Run \verb|make ${DOCNAME}.html| and briefly review the produced
HTML file.  Warnings or even error messages in this process are
(regrettably) still to be expected, so at least superficial human inspection 
of the output is necessary.
\item Edit the Makefile and update the DOCVERSION, DOCDATE, and DOCTYPE
variables according to the new document release.
\item Run \verb|make package|.  If no major problems become visible,
\item run \verb|svn commit| (or the equivalent for your VCS) -- this
will update the source if VCS information is represented as per
section~\ref{sect:vcs}.
\item Run \verb|make upload|.  This will ask you to review the document
metadata in a python dictionary, let you add a note for the document
coordinator and performs the upload.
\item Review the file \texttt{docrepo-response.html} created in the
document directory; this contains the raw HTML the uploading script got
back from the document repository.  We do not automatically check for
errors yet, so again human attention is required.
\end{enumerate}


\section{Customisation and Development}
\label{sect:impl}

This section discusses aspects of \ivoatex\ that are more technical in
nature.  Authors with a modicum of \TeX\ expertise are nevertheless
encouraged to read it.

\subsection{Technical Overview}

The central files in \ivoatex\ processing are

\begin{bigdescription}
\item[ivoa.cls] The class file, inheriting from \LaTeX's article class.
The file defines the markup rules for PDF processing, including
titlepage generation and extra macros and environments.  Its content is
ignored for HTML generation.

\item[tthdefs.tex] This file protects its contents from normal \TeX\
processing by a \verb|\iftth| conditional. This way, only tth sees
definitions made here. Each special feature defined in \texttt{ivoa.cls}
has a counterpart here, giving rules for translation to HTML.  This
usually encompasses emitting some HTML before and after the argument of
a TeX construct, where material between \verb|\begin{html}| and
\verb|\end{html}| is included literally in the HTML document.

\item[tth-ivoa.xslt] An XSLT stylesheet that postprocesses tth's output
and performs some operations that would be inconvenient to implement in
\texttt{tthdefs.tex}, in particular for the formatting of the opening
material.

\item[Makefile] This makefile is included by the user makefile in the
document directory proper.  It defines the rules given above as well as
some extra housekeeping rules like package building and building tth
from its source.

\end{bigdescription}

\subsection{Semantic Markup}

In order to make it support rich, semantic markup, \ivoatex\ needs to be
continuously developed.  In particular, it is good practice to define
macros for marking up values of certain datatypes, as with \ivoatex's
\texword{xmlel} and \texword{vorent}.  
Thus, whenever a document has multiple
instances of such values, authors should define macros and use these.
For instance, RegTAP deals with lots of concepts from its own
database schema and hence has
\begin{lstlisting}[language=TeX]
\definecolor{rtcolor}{rgb}{0.15,0.4,0.3}
\newcommand{\rtent}[1]{\texttt{\color{rtcolor} #1}}
\end{lstlisting}
in its document preamble to
define markup for ``RegTAP entity'', whereas
this note, as it mentions many words with a special meaning to \TeX, has
\begin{lstlisting}[language=TeX]
\definecolor{texcolor}{rgb}{0.4,0.1,0.1}
\newcommand{\texword}[1]{\texttt{\color{texcolor} #1}}
\end{lstlisting}
Such macros will be included in \ivoatex\ itself rather than an
individual document's preamble when they prove useful in multiple
documents.

\subsection{Custom Macros and Environments}

The tth translator used by \ivoatex\ ignores \texword{usepackage}.  Many
common packages are natively supported, but those that are not in
general need specific handling, and sometimes support is somewhat spotty.  
For instance, the \texword{nolinkurl}
macro is not supported natively by tth, and hence in
\texttt{tthdefs.tex} there is code to the effect of
\begin{lstlisting}[language=TeX,basicstyle=\footnotesize]
\newcommand{\nolinkurl}[1]{%
  \special{html:<span class='nolinkurl'>}#1\special{html:</span>}}
\end{lstlisting}

When a document requires special markup, it is likely that 
different implementations will be necessary for PDF and HTML output.
Using \texword{iftth} the implementations for the current output mode
can be selected (without the \texword{newif} mentioned in the tth
documentation, as that is already performed in \texttt{tthdefs.tex}).

For instance, RegTAP 1.0 had many inline tables that need special spacing
the PDF rendering, whereas normal tables will do for them
in HTML.  It therefore
had in its preamble the definitions
\begin{lstlisting}
\iftth
  \newenvironment{inlinetable}{}{}
\else
  \newenvironment{inlinetable}{\vskip 1ex\vfil
    \penalty8000\vfilneg%
    \hbox to\hsize\bgroup\hss}
  {\hss\egroup\vspace{8pt}}
\fi
\end{lstlisting}

(this mechanism proved useful for other specifications, too, and so
it is part of \ivoatex~proper now).

\subsection{Custom CSS}
\label{sect:customcss}

If you find you need custom CSS to fix HTML formatting, you should
probably talk to \ivoatex's authors first.  There are, however, 
legitimate cases when something needs extra styling in HTML that 
comes out right without further effort
in the PDF output.  In such cases, a custom CSS file can
be added to a repository (it must then also be added to \texttt{SOURCES}
in the Makefile in order for it to be delivered with the document
package).

The document itself would then use the \ivoatex's \texword{customcss}
macro in its preamble with the CSS file name as an argument.  For
example, the source for this document says

\begin{lstlisting}
\iftth
 \newcommand{\comicstuff}[1]{
    \begin{html}<span class="comic">#1</span>\end{html}}
\else
  \newcommand{\comicstuff}[1]{(HTML exclusive material)}
\fi
\end{lstlisting}

\noindent in its preamble.  With this (and a CSS file that can be inspected in the
distribution),

\begin{lstlisting}
\comicstuff{If this is comic sans, your web browser is permissive.}
\end{lstlisting}

\noindent becomes: 
\comicstuff{If this is comic sans, your web browser is permissive.}

\subsection{Migration from Ivoadoc}

To ease migration from documents authored in ivoadoc, \ivoatex\ comes
with an XSLT stylesheet writing out a starting point for an ivoatex
version.  While many desirable features (e.g., extraction of titlepage
information) are not implemented and translated tables are incomplete,
the styleheet should still save time.  For XHTML-compliant ivoadoc
sources, the stylesheet is used like this:

\begin{lstlisting}[language=sh]
xsltproc ivoatex/fromivoadoc.xslt ivoadoc_source.html
\end{lstlisting}

\subsection{Maintenance of the Architecture Diagram}

The IVOA architecture diagram is introduced in \citet{note:VOARCH} to
visualise the standards landscape.  All IVOA recommendations should have
an architecture diagram showing the current standard as well as the
related standards within that landscape.

Within \ivoatex, architecture diagrams are produced in Scalable Vector
Graphics (SVG).  The source is in \texttt{ivoatex/archdiag-full.xml},
specifying the location of the recommendations (in \xmlel{rec} elements)
and the documents on the recommendation track (in \xmlel{prerec}
elements).  The figure has a design size of $800\times 600$ ``pixels''.
Note that SVG is not really pixel-based -- the numbers are just
convenient, unitless floating point numbers, and the conversion of these
coordinates to physical ones is done at render time.

In the diagram, the standards shoud be limited to the inner box, i.e.,
the zone between 50, 100 and 750, 500.  Standards boxes are $90\times
18$ pixels large, which is set in the \texttt{format-standard} template
in \texttt{make-archdiag.xslt}.  When standard boxes are aligned, the
vertical distance of their centers should be 25 pixels, the horizontal
distance 100 pixels.  To keep the diagram lively, standards not
obviously grouped with others may be placed ``off-grid''.

The specifications in \texttt{archdiag-full.xml}, as well as those in
the authors' \texttt{role\_diagram.xml}, are ad-hoc XML interpreted by the
stylesheet \texttt{ivoatex/make-archdiag.xml}.  That stylesheet is
written such that all three levels of the architecture diagram can be
created.  The \texttt{Makefile} in \texttt{ivoatex} has the necessary
rules, but in contrast to the author rules, they are expected to be
executed \emph{within} the ivoatex directory.

So, to create the full versions of the three levels of the architecture
diagram, do something like

\begin{lstlisting}[language=sh]
cd ivoatex
make archdiag-l0.svg
make archdiag-l1.svg
make archdiag-l2.svg
\end{lstlisting}

The resulting svg files can be viewed in common browsers or in vector
graphics programs like inkscape.  The latter can also be used to find
good positions for new elements (cursor coordinates are shown in the
footline, but the direction of the y axis is reversed versus the SVG
coordinates).  Please do not use graphical tools to edit the diagram
itself -- the goal of the current architecture is to make edits
transparent and have a clear and simple specification of the standards
themselves in a file producing meaningful diffs.

\section{Desirable Features to be Implemented}

A major drawback of \ivoatex's HTML output is that paragraphs are not actually
marked up as such.  Due to the \TeX\ processing model, their
reconstruction is non-trivial.  Hence in the generated HTML,
source-level paragraphs are rendered as text nodes separated by empty
HTML paragraph elements.  It would probably be possible to rectify this
in the XSLT postprocessing.

An automated way to maintain the in-document history (i.e., the sequence
of \verb|\previousversions| in the preamble) and better support to
generate the change log would be desirable.

The hardcoded and part-manual way to run \LaTeX and \BibTeX~should be
made smarter, perhaps along the lines of latex-make.

\appendix
\section{Changes from Previous Versions}

\subsection{Changes from Version 1.1}

\begin{itemize}
\item Added material on the SVG architecture diagram, removed references
to old PNG-based workflow.
\item Added an upload checklist (and, relatedly, listing legal WG names).
\item Added documentation on auxiliaryurl.
\item Added instructions on making standards records.
\item Documented triple-bibliography maintenance.
\end{itemize}

\subsection{Changes from Version 1.0}
\begin{itemize}
\item Changed google code URLs to volute.g-vo.org ones.
\item Documented new facilities for generating material, with extra
focus on auto-documenting XML schema.
\item Added advice on citing IVOA recommendations.
\item Re-targeting for \ivoatex 1.0 (rather than 0.4 before)
\end{itemize}


\bibliography{ivoatex/ivoabib,ivoatex/docrepo}


\end{document}
